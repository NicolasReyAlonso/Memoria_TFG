\chapter{Introducción}
\section{Motivación}
Al iniciar mi periodo de prácticas en la empresa \gls{itc}, me encontré con la tarea
de desarrollar una librería frontend para facilitar el desarrollo de aplicaciones web
basadas en el marco \gls{nextgendem}. Este proyecto, denominado \gls{nextgendem-gui-lib},
tenía como objetivo principal acelerar el proceso de desarrollo frontend al proporcionar
una serie de componentes reutilizables y funcionalidades predefinidas. Durante el desarrollo
de esta librería, me di cuenta de que, aunque era útil para proyectos específicos,
su aplicabilidad estaba limitada debido a su dependencia del backend existente.

Tras llegar a esta realización y siendo consciente de las limitaciones temporales de mis
 prácticas, consulté con mi tutor en la empresa la posibilidad de ampliar el alcance del
  proyecto para incluir también un framework backend, algo que se me había sugerido
  desde el comienzo del desarrollo, solo que dada la escasez de tiempo decidí no priorizar
  para entregar un software funcional. La idea era crear un conjunto completo de 
  herramientas que permitiera el desarrollo rápido de aplicaciones web, tanto en el frontend
  como en el backend. Esta expansión, requería de tiempo adicional, lo que nos ha llevado
  a plantearlo como TFG. Asimismo, creo que este desarrollo será de gran valor para mi 
  formación, proporcionandome un reto técnico significativo y la oportunidad de aplicar
  mis conocimientos en un proyecto real y completo.
\subsection{Contexto}
Breve contextualización del problema.
\subsubsection{Definición del problema}
Descripción del problema tratado.
\paragraph{Limitaciones}
Limitaciones y alcance.
\subparagraph{Notas}
Notas adicionales.