% Plantilla TFG — Nicolás Rey Alonso
% Proyecto: Un framework backend Python (FastAPI + Docker) y frontend Angular
\documentclass[12pt,a4paper]{book}
\usepackage[utf8]{inputenc}
\usepackage[T1]{fontenc}
\usepackage[spanish]{babel}
\usepackage{graphicx}
\usepackage{xcolor}
\usepackage{geometry}
\usepackage{fancyhdr}
\usepackage{hyperref}
\usepackage{titlesec}
\usepackage{lipsum}
\usepackage{booktabs}
\usepackage{enumitem}
\usepackage{caption}

% Márgenes
\geometry{left=3cm,right=2.5cm,top=3cm,bottom=3cm}

% Niveles de numeración y TOC hasta subparagraph
\setcounter{secnumdepth}{5}
\setcounter{tocdepth}{5}

% Hipervínculos
\hypersetup{
  pdftitle={TFG — Un framework backend Python y frontend Angular},
  pdfauthor={Nicolás Rey Alonso},
  colorlinks=true,
  linkcolor=blue!65!black,
  citecolor=blue!65!black,
  urlcolor=blue!65!black
}

% Encabezado con logo (Encabezado) en todas las páginas
\pagestyle{fancy}
\fancyhf{}
\setlength{\headheight}{48pt}
\fancyhead[L]{\raisebox{-0.4\baselineskip}{\includegraphics[height=1.6cm]{Media/ULPGC/logo_Encabezado.png}}}
\fancyhead[R]{\thepage}
\renewcommand{\headrulewidth}{0.4pt}

% Título y portada
\begin{document}

% --- Portada ---
\begin{titlepage}
  \centering
  \vspace*{1.5cm}
  \includegraphics[width=0.45\textwidth]{Media/ULPGC/logo_Portada}\par
  \vspace{1.2cm}
  {\scshape\LARGE Universidad de Las Palmas de Gran Canaria \par}
  \vspace{1cm}
  {\huge\bfseries Framework backend y Frontend \\ para el desarrollo rápido\par}
  \vspace{1cm}
  {\Large Trabajo Fin de Grado\par}
  \vfill
  {\Large Autor: \\\textbf{Nicolás Rey Alonso}\par}
  {\Large Grado en Ingeniería Informática — 4º\par}
  \vspace{0.8cm}
  {\Large ULPGC\par}
  \vspace{1.5cm}
  {\Large Fecha: Abril 2026\par}
  \thispagestyle{empty}
\end{titlepage}

% Índice (con numeración romana en frontmatter)
\frontmatter
\cleardoublepage
\tableofcontents
\cleardoublepage

% Opcional: listas de figuras/ tablas
% \listoffigures
% \listoftables

\mainmatter

\chapter{Resumen}
\addcontentsline{toc}{chapter}{Resumen}
Resumen (escribe aquí un resumen breve del TFG).

\chapter{Introducción}
\section{Motivación}
Explica la motivación del proyecto.
\subsection{Contexto}
Breve contextualización del problema.
\subsubsection{Definición del problema}
Descripción del problema tratado.
\paragraph{Limitaciones}
Limitaciones y alcance.
\subparagraph{Notas}
Notas adicionales.

\chapter{Tecnologías y marco teórico}
\section{FastAPI}
Descripción y por qué se eligió.
\section{Docker}
Uso en despliegue y desarrollo.
\section{Angular}
Motivos para usar Angular en el frontend.

\chapter{Arquitectura del sistema}
\section{Estructura del backend}
Detalles del framework, módulos, API.
\section{Estructura del frontend}
Componentes principales y flujo de datos.

\chapter{Desarrollo y despliegue}
\section{Desarrollo rápido}
Buenas prácticas, hot-reload, scaffolding.
\section{Contenedores y CI/CD}
Cómo dockerizar la aplicación y ejemplos de pipeline.

\chapter{Resultados}
\section{Evaluación}
Pruebas y rendimiento.

\chapter{Conclusiones y trabajo futuro}
\section{Conclusiones}
\section{Trabajo futuro}

\appendix
\chapter{Instalación rápida}
Instrucciones para ejecutar el proyecto localmente:
\begin{enumerate}
  \item Clonar el repositorio.
  \item Backend: construir imagen Docker y ejecutar (ej: `docker compose up --build`).
  \item Frontend: `npm install` y `ng serve`.
\end{enumerate}

\backmatter
\bibliographystyle{plain}
\bibliography{bibliografia}


\end{document}
