% Plantilla TFG — Nicolás Rey Alonso
% Proyecto: Un framework backend Python (FastAPI + Docker) y frontend Angular
\documentclass[12pt,a4paper,oneside]{book}
\usepackage[utf8]{inputenc}
\usepackage[T1]{fontenc}
\usepackage[spanish]{babel}
\usepackage{graphicx}
\usepackage{xcolor}
\usepackage{geometry}
\usepackage{fancyhdr}
\usepackage{hyperref}
\usepackage{titlesec}
\usepackage{lipsum}
\usepackage{booktabs}
\usepackage{enumitem}
\usepackage{caption}
\usepackage[acronym]{glossaries}
\makeglossaries
\newacronym{itc}{ITC}{Instituto Tecnológico de Canarias}
% Márgenes
\geometry{left=3cm,right=2.5cm,top=3cm,bottom=3cm}

% Niveles de numeración y TOC hasta subparagraph
\setcounter{secnumdepth}{5}
\setcounter{tocdepth}{5}

% Hipervínculos
\hypersetup{
  pdftitle={TFG — Un framework backend Python y frontend Angular},
  pdfauthor={Nicolás Rey Alonso},
  colorlinks=true,
  linkcolor=blue!65!black,
  citecolor=blue!65!black,
  urlcolor=blue!65!black
}

% Encabezado con logo (Encabezado) en todas las páginas
\pagestyle{fancy}
\fancyhf{}
\setlength{\headheight}{48pt}
\fancyhead[L]{\raisebox{-0.4\baselineskip}{\includegraphics[height=1.6cm]{Media/ULPGC/logo_Encabezado.png}}}
\fancyhead[R]{\thepage}
\renewcommand{\headrulewidth}{0.4pt}

% Mantener el estilo fancy en todas las secciones
\fancypagestyle{plain}{%
  \fancyhf{}
  \fancyhead[L]{\raisebox{-0.4\baselineskip}{\includegraphics[height=1.6cm]{Media/ULPGC/logo_Encabezado.png}}}
  \fancyhead[R]{\thepage}
  \renewcommand{\headrulewidth}{0.4pt}
}

% Título y portada
\begin{document}

\frontmatter
\begin{titlepage}
  \centering
  \vspace*{1.5cm}
  \includegraphics[width=0.45\textwidth]{Media/ULPGC/logo_Portada}\par
  \vspace{1.2cm}
  {\scshape\LARGE Universidad de Las Palmas de Gran Canaria \par}
  \vspace{1cm}
  {\huge\bfseries Framework backend y Frontend \\ para el desarrollo rápido\par}
  \vspace{1cm}
  {\Large Trabajo Fin de Grado\par}
  \vfill
  {\Large Autor: \\\textbf{Nicolás Rey Alonso}\par}
  {\Large Grado en Ingeniería Informática — 4º\par}
  \vspace{0.8cm}
  {\Large ULPGC\par}
  \vspace{1.5cm}
  {\Large Fecha: Abril 2026\par}
\end{titlepage}
\include{indice}
\mainmatter
\chapter{Prefacio}
\section{Origen del Desarrollo}
Este proyecto surge como una expansión del desarrollo ejercido en mis prácticas de empresa.
 ``\textbf{NextGenDem-GUI-LIB}'' fue diseñado originalmente para facilitar y acelerar el
 desarrollo frontend en base a un backend específico. 
 
 Este proyecto, aunque útil, limitaba su aplicabilidad
 a proyectos con arquitecturas idénticas y requería en el caso de que se deseasen
 realizar modificaciones en el backend de reescribir mucho código. 
 Reconociendo esta limitación y consultándolo con mi tutor de empresa \gls{itc}, se decidió
ampliar el alcance del proyecto para crear un framework completo que incluyese tanto
backend como frontend, permitiendo así su uso en una variedad más amplia de proyectos y
arquitecturas.

Por otra parte, el desarrollo del proyecto ya estaba avanzado y debido a las Limitaciones
temporales de las prácticas de empresa tuve que redirigir mi desarrollo hacia un
punto intermedio, ya que este framework se deseaba utilizar para el desarrollo de un proyecto
nuevo en la empresa. Así, el proyecto original se orientó a crear un framework que facilitase el 
desarrollo rápido de aplicaciones web. En base a este desarrollo se planteó el presente
Trabajo Fin de Grado, con el objetivo de documentar y presentar el framework completamente
desarrollado, incluyendo tanto el backend como el frontend, y demostrar su utilidad
en el desarrollo rápido de aplicaciones web.

\section{Estado del desarrollo}
El desarrollo del framework se encuentra en una fase intermedia. El frontend tiene
una buena parte de sus funcionalidades implementadas, incluyendo componentes
básicos y algunas funcionalidades avanzadas. Sin embargo, aún quedan aspectos
por completar y optimizar.

El backend, por otro lado, está en una etapa más temprana de desarrollo aunque se me ha
dado una base bastante completa. Esto se debe a que el \gls{itc} requería de un backend
funcional para el desarrollo de su ultimo proyecto, por lo que se priorizó su desarrollo
ya que el tiempo era limitado. A pesar de esto, el backend aún necesita mejoras y
adiciones para alcanzar su pleno potencial.

En resumen, el framework no está completamente terminado y requiere de más trabajo
para ser considerado finalizado. Es aquí dondo empiezo mi desarrollo en el presente
Trabajo Fin de Grado, con el objetivo de completar, optimizar y terminar tanto el backend
como el frontend del framework.

\section{Planificación previa al desarrollo}
Antes de iniciar el desarrollo del framework, se realizó una pseudoplanificación con
el objetivo de establecer una hoja de ruta clara y estructurada para la realización 
de una planificación, esta vez, completa del proyecto.

\subsection{Frontend}
Debido a que el desarrollo del frontend fue realizado durante mis prácticas de empresa 
por mi, soy plenamente conocedor del alcance y las tareas necesarias para su
finalización.

\subsection{Backend}
El backend, al ser un desarrollo que no realicé yo, no tenía un conocimiento
tan profundo de su alcance y las tareas necesarias para su finalización. Por ello,
contacté con mi tutor de empresa en el \gls{itc} para obtener una visión clara
de las tareas pendientes y los objetivos a alcanzar.

\chapter{Introducción}
\section{Motivación}
Explica la motivación del proyecto.
\subsection{Contexto}
Breve contextualización del problema.
\subsubsection{Definición del problema}
Descripción del problema tratado.
\paragraph{Limitaciones}
Limitaciones y alcance.
\subparagraph{Notas}
Notas adicionales.

\chapter{Tecnologías y marco teórico}
\section{FastAPI}
Descripción y por qué se eligió.
\section{Docker}
Uso en despliegue y desarrollo.
\section{Angular}
Motivos para usar Angular en el frontend.

\chapter{Arquitectura del sistema}
\section{Estructura del backend}
Detalles del framework, módulos, API.
\section{Estructura del frontend}
Componentes principales y flujo de datos.

\chapter{Desarrollo y despliegue}
\section{Desarrollo rápido}
Buenas prácticas, hot-reload, scaffolding.
\section{Contenedores y CI/CD}
Cómo dockerizar la aplicación y ejemplos de pipeline.

\chapter{Resultados}
\section{Evaluación}
Pruebas y rendimiento.

\chapter{Conclusiones y trabajo futuro}
\section{Conclusiones}
\section{Trabajo futuro}

\appendix
\chapter{Instalación rápida}
Instrucciones para ejecutar el proyecto localmente:
\begin{enumerate}
  \item Clonar el repositorio.
  \item Backend: construir imagen Docker y ejecutar (ej: `docker compose up --build`).
  \item Frontend: `npm install` y `ng serve`.
\end{enumerate}

\backmatter
\bibliographystyle{plain}
\bibliography{bibliografia}
\printglossaries

\end{document}
