\documentclass[oneside,12pt]{book}
\usepackage[utf8]{inputenc}
\usepackage{standalone}
\usepackage[margin=2.5cm]{geometry}
\usepackage{graphicx}
\usepackage{hyperref}
\usepackage{lmodern}
\usepackage{float}
\usepackage[spanish]{babel}
\usepackage{csquotes}
\usepackage[figurename=Ilustración]{caption}
\usepackage[numbers]{natbib}
\usepackage{ragged2e}
\usepackage{array}
\usepackage{multirow}
\usepackage{parskip}
\usepackage{listings}
\usepackage{epigraph}
\usepackage{lipsum} % Texto de relleno en plantilla

\usepackage{algorithm}
\usepackage{algpseudocode}

% Glosario de términos - BEGIN
\usepackage{glossaries}

\makeglossaries

\newglossaryentry{latex}
{
    name=latex,
    description={Is a markup language specially suited 
    for scientific documents}
}

\newglossaryentry{maths}
{
    name=mathematics,
    description={Mathematics is what mathematicians do}
}
% Glosario de términos - END

%\usepackage{fancyhdr}
\usepackage{geometry}
\usepackage{color}\usepackage{graphicx,xcolor}

\renewcommand{\lstlistingname}{Algoritmo}
%\renewcommand{\lstlistlistingnam%e}{Índice de~\lstlistingname s}

\floatname{algorithm}{Algoritmo}
\renewcommand{\listalgorithmname}{Índice de~\lstlistingname s}


\lstset{
        basicstyle=\fontsize{7}{11}\ttfamily,
        language=Java,
        captionpos=b,
}

\usepackage{titlesec}
\setcounter{secnumdepth}{3}
\titleformat{\chapter}[display]
{\normalfont\huge\bfseries}{\chaptertitlename\ \thechapter}{20pt}{\Huge}

\urlstyle{same}
\title{Título aquí}
\author{Nombre Estudiante}
\date{Curso 20XX - 20XX}
\renewcommand{\contentsname}{Contenido}
\renewcommand{\figurename}{Ilustración}
\usepackage{pifont}
\renewcommand\labelitemi{\ding{52}}

\begin{document}
%\maketittle
\begin{titlepage}

\begin{center}

\vspace*{0.25in}
\begin{figure}[htb]
\begin{center}
\includegraphics[width=15cm]{Ilustraciones/NuevoLogoEII.png}
%\includegraphics[width=15cm]{Ilustraciones/LogoEII.jpg}
%\includegraphics[width=20cm]{Ilustraciones/Logo_EII+ULPGC.png}
\end{center}
\end{figure}

\vspace*{0.25in}
\vspace*{0.25in}
\begin{Huge}
    \textbf{Trabajo de Fin de Grado} \\
\end{Huge}
\vspace*{0.5in}

\noindent\hfil\rule{17cm}{0.2mm}\hfil\\

\vspace*{0.1in}
\begin{Huge}
    \textbf{Framework Backend–Frontend para el desarrollo rápido de proyectos apoyado en un modelo de datos en evolución: aplicación a los proyectos con el Jardín Botánico Viera y Clavijo} \\
\end{Huge}
\vspace*{0.3in}
\begin{large}
TITULACIÓN: Grado en Ingeniería Informática \\
\vspace*{0.1in}
AUTOR: Nicolás Rey Alonso \\
\end{large}
\vspace*{0.3in}
\noindent\hfil\rule{17cm}{0.2mm}\hfil\\
\vspace*{0.1in}
\begin{large}
TUTORIZADO POR: \\
Rafael Juan Nebot Medina \\
María Dolores Afonso Suárez \\
\end{large}
\vspace*{0.3in}
Febrero de 2026

\end{center}


\end{titlepage}


\newgeometry{margin=1.5cm}

\thispagestyle{empty}

%\lhead{ 
\begin{tabular}{p{15cm}p{5cm}}
  %\includegraphics{Ilustraciones/LogoEII.jpg}   &  TFT04\\
  \includegraphics[width=7cm]{Ilustraciones/NuevoLogoEII.png}   &  TFT04\\
\end{tabular}
%}


\fboxrule=2pt
\begin{center}
\fcolorbox{gray}{gray!10}{\parbox{.8 \textwidth}{ {\large SOLICITUD DE DEFENSA DE TRABAJO DE FIN DE TÍTULO}}}
\end{center}


\justify
D./Dª Nicolás Rey Alonso, autor del trabajo Framework Backend–Frontend para el desarrollo rápido de proyectos apoyado en un modelo de datos en evolución: aplicación a los proyectos con el Jardín Botánico Viera y Clavijo, correspondiente a la titulación Grado de Ingeniería Informática, en colaboración con la empresa ITC (Instituto Tecnológico de Canarias).

\vspace{1em}
SOLICITA

\vspace{1em}
que se inicie el procedimiento de defensa del mismo, para lo que se adjunta la documentación requerida, haciendo constar que 

[X] se autoriza / [ ] no se autoriza la grabación en audio de la exposición y turno de preguntas.

\vspace{1em}
Asimismo, con respecto al registro de la propiedad intelectual/industrial del TFT, declara que:

[ ] Se ha iniciado o hay intención de iniciarlo (defensa no pública).

[ ] No está previsto.

\vspace{1em}
Y para que así conste firma la presente. (fecha en firma electrónica)

\begin{center}
%En Las Palmas de Gran Canaria, a día de mes de año

\vspace{1em}
El/La estudiante

\vspace{3em}
Fdo.------------------------
\end{center}

\begin{center}
\fcolorbox{gray}{white!10}{\parbox{.9\textwidth}{
{\tiny A rellenar y firmar \textbf{obligatoriamente} por el/la/los/las tutores}

En relación con la presente solicitud, se informa:{\tiny (firmar donde corresponda)}

\vspace{1em}
\begin{center}
\begin{tabular}{|p{\dimexpr.45\linewidth}|p{\dimexpr.45\linewidth}|}
%\begin{tabular}{|l|r|}
    \hline
   & \\
  Positivamente {\tiny (en  caso  de  detección  de  copia, esta firma quedará invalidada)}  & Negativamente {\tiny (justificación en TFT05)} \\
   & \\
   & \\
   & \\
   & \\
   & \\
   & \\
  Fdo.------------------------ & Fdo.------------------------  \\
     & \\

    \hline
\end{tabular} 

\end{center}

\vspace{1em}

}
}



%\cfoot{
DIRECTOR DE LA ESCUELA DE INGENIERÍA INFORMÁTICA
%}

\end{center}

\restoregeometry

\clearpage

\newpage
\pagenumbering{roman}
{\Large{\textbf{Agradecimientos}}}

\begin{flushright}
    {\setlength{\parskip}{8mm}
        \large{
            \textit{Al ITC por concederme la oportunidad de participar en este proyecto}
        }
    }
    
\end{flushright}

\clearpage
\input{Resumen.tex}
\tableofcontents

%\addcontentsline{toc}{section}{Listings}
%\lstlistoflistings
\listofalgorithms 

\listoffigures

\listoftables


\clearpage
\pagenumbering{arabic}

\chapter{Introducción}
\chapter{Introducción}
%Préambulo de la historia de las aplicaciones Android (?)
\epigraph{``Los buenos programadores saben qué escribir. Los grandes saben qué reutilizar.''}{Eric S. Raymond \cite{raymond1999cathedral}}
%\lipsum[2]
En la actualidad el software tiene tres requisitos fundamentales: Bajo costo, alta seguridad, alta eficiencia y bajo tiempo de desarrollo. El ritmo frenético de la evolución de las tecnologías requiere de la capacidad de generar software de alta calidad en muy poco tiempo. Esto, en ocasiones supone una antítesis, ya que para refinar y probar correctamente un producto es necesaria la inversión de tiempo.

Dada las necesidades del mercado se han creado herramientas que permiten esto: \gls{framework}s. Estos productos de software empaquetan las funcionalidades necesarias para el desarrollo rápido, permitiendo asegurar y \gls{testear} la base del código asegurando productos firmes, de bajo costo y alta calidad. Sin embargo, a pesar de ser software muy bueno, los frameworks comerciales tienen limitaciones ya que es imposible crear una solución universal que satisfaga las necesidades individuales de cada desarrollo, por lo que es necesaria siempre la introducción de arquitecturas de software que, con ayuda de la herramienta, completen los requisitos del proyecto.

En este punto es donde incurre el presente tfg, desarrollando herramientas añadidas y funcionalidades para completar un framework que se ajuste a las arquitecturas y desarrollos que realice el \gls{itc}.


Ejemplo para el glosario de términos (ver al final del documento):

This final report has been produced using \Gls{latex}, a tool specially suitable for technical documents that include \gls{maths}. 

\chapter{Estado actual y objetivos iniciales}
%\input{EstadoDelArte.tex}
Este es un CAPÍTULO \textbf{OBLIGATORIO}.

\section{Motivación y antecedentes}

Descripción de la motivación inicial y el estado actual en la temática concreta del trabajo.

\section{Objetivos}

Desglose de los \textbf{objetivos} inicialmente previstos en el TFT01. Opcionalmente, pueden adelantarse las desviaciones que hayan podido producirse con respecto a esos objetivos iniciales.

NOTA: deben evitarse expresiones con referencias directas a formularios del tipo ``en el TFT01 ...'', puesto que cuando la memoria sea consultada fuera del entorno de la EII no serán identificables. Deben sustituirse por ``en la propuesta inicial ...''.

\chapter{Aportaciones del trabajo}
\chaptermark{Competencias, aportaciones y ODS}
%\input{Competencias.tex}
Este es un CAPÍTULO \textbf{OBLIGATORIO}.

También pueden incluirse parte de estos contenidos en las conclusiones, en la descripción de los objetivos o en forma de anexos.

\section{Principales aportaciones}

Justificar qué es lo que este TFT \textbf{aporta} a nuestro entorno socio-económico, técnico o científico. Repercusión esperada.

\section{Competencias específicas}

Indicar, sólo para las \textbf{competencias específicas} relacionadas de forma más directa con el trabajo desarrollado, cómo se han cubierto con este TFT.

\section{Alineamiento con los objetivos de desarrollo sostenible}

\begin{table}[!htbp]
\caption{Grado de relación del TFT con los objetivos de desarrollo sostenible.}
\label{tab:ODS}
\centering
\begin{tabular}{|l|c|c|c|c|}
\cline{1-5}
  & \multicolumn{4}{c|}{Grado de relación} \\ \cline{2-5} 
 ODS & 0 & 1 & 2 & 3 \\
  & No procede & Bajo & Medio & Alto \\  
  \cline{1-5} 
1 Fin de la Pobreza & & & & \\ \cline{1-5}
2 Hambre cero & & & & \\
\cline{1-5}
3 Salud y Bienestar & & & & \\
\cline{1-5}
4 Educación de calidad & & & & \\
\cline{1-5}
5 Igualdad de género & & & & \\
\cline{1-5}
6 Agua limpia y saneamiento & & & & \\
\cline{1-5}
7 Energía Asequible y no contaminante & & & & \\
\cline{1-5}
8 Trabajo decente y crecimiento económico & & & & \\
\cline{1-5}
9 Industria, Innovación e Infraestructuras & & & & \\
\cline{1-5}
10 Reducción de las desigualdades & & & & \\
\cline{1-5}
11 Ciudades y comunidades sostenibles & & & & \\
\cline{1-5}
12 Producción y consumo sostenibles & & & & \\
\cline{1-5} 
13 Acción por el clima & & & & \\
\cline{1-5}
14 Vida submarina & & & & \\
\cline{1-5}
15 Vida de ecosistemas terrestres & & & & \\
\cline{1-5}
16 Paz, justicia e instituciones sólidas & & & & \\
\cline{1-5}
17 Alianzas para lograr objetivos & & & & \\
\cline{1-5}
\end{tabular}%
\end{table}

Justificar el alineamiento del TFT con los \textbf{ODS} con los que presente un mayor grado de relación, en función de lo indicado en la tabla~\ref{tab:ODS}.

\chapter{Desarrollo}
%\input{Desarrollo.tex}
Este es un CAPÍTULO \textbf{OBLIGATORIO}.

\section{Metodología}

Descripción de la \textbf{Metodología} aplicada a lo largo del trabajo y justificación de la elección.

\section{Fases de desarrollo}

Desarrollo del trabajo en sus distintas fases (por ejemplo, para una metodología en cascada: análisis, diseño, implementación, validación). Deberán comentarse especialmente las decisiones de diseño tomadas y la selección de las herramientas empleadas.

Si el trabajo incluye software desarrollado, deberán seleccionarse las secciones más relevantes del mismo y comentarlas en la memoria.

Ajuste a la planificación inicialmente prevista.

Modificación en los objetivos planteados.

\chapter{Resultados}
%\input{Resultados.tex}

Presentación de los resultados del trabajo. Repercusión.

\chapter{Conclusiones y trabajo futuro}
%\input{Conclusiones.tex}
Este es un CAPÍTULO \textbf{OBLIGATORIO}.

\section{Conclusiones}

Valoración de resultados, grado de consecución de los  objetivos.

\section{Trabajo futuro}

Líneas de trabajo futuro, aspectos pendientes, posibles extensiones.

\section{Uso de la IA}

Indicar el uso que se ha hecho de la IA tanto en la elaboración de este documento como en el desarrollo del TFG.

\vspace{1cm}

\textbf{EXTENSIÓN DE LA MEMORIA:}

\begin{itemize}
    \item GII, GIFM, DGII-ADE (parte de GII): entre 50 y 100 páginas
    \item GCID: entre 75 y 100 páginas
\end{itemize}

\chapter{Otros capítulos y anexos}

El documento debe terminar con las \textbf{referencias bibliográficas} (SECCIÓN \textbf{OBLIGATORIA}). Después pueden incluirse \textbf{anexos} de forma opcional.

\section{Ejemplos de capítulos opcionales}

Dependiendo del tipo de trabajo, se podrían incluir, en el orden que corresponda, capítulos adicionales como los siguientes:

\begin{description}
\item{REQUISITOS}
\item{NORMATIVA Y LEGISLACIÓN} incluir la legislación vigente que afecte al TFT (ley de protección de datos, leyes sobre seguridad, ...)
\item{ASPECTOS ECONÓMICOS Y TEMPORALES}
\item{DISEÑO}
\item{RESULTADOS EXPERIMENTALES}
\item{MANUAL  DE  USUARIO  Y  SOFTWARE}  deberán  incluirse  obligatoriamente  en  la  memoria  los extractos más relevantes   del código desarrollado. Siempre que sea posible, deberá   proporcionarse acceso a un repositorio software.
\end{description}


\bibliography{ref}
\bibliographystyle{apalike}
\clearpage

\printglossaries

\end{document}