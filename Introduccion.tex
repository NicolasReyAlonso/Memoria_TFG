\chapter{Introducción}
%Préambulo de la historia de las aplicaciones Android (?)
\epigraph{``Los buenos programadores saben qué escribir. Los grandes saben qué reutilizar.''}{Eric S. Raymond \cite{raymond1999cathedral}}
%\lipsum[2]
En la actualidad el software tiene tres requisitos fundamentales: Bajo costo, alta seguridad, alta eficiencia y bajo tiempo de desarrollo. El ritmo frenético de la evolución de las tecnologías requiere de la capacidad de generar software de alta calidad en muy poco tiempo. Esto, en ocasiones supone una antítesis, ya que para refinar y probar correctamente un producto es necesaria la inversión de tiempo.

Dada las necesidades del mercado se han creado herramientas que permiten esto: \gls{framework}s. Estos productos de software empaquetan las funcionalidades necesarias para el desarrollo rápido, permitiendo asegurar y \gls{testear} la base del código asegurando productos firmes, de bajo costo y alta calidad. Sin embargo, a pesar de ser software muy bueno, los frameworks comerciales tienen limitaciones ya que es imposible crear una solución universal que satisfaga las necesidades individuales de cada desarrollo, por lo que es necesaria siempre la introducción de arquitecturas de software que, con ayuda de la herramienta, completen los requisitos del proyecto.

En este punto es donde incurre el presente tfg, desarrollando herramientas añadidas y funcionalidades para completar un framework que se ajuste a las arquitecturas y desarrollos que realice el \gls{itc}.
