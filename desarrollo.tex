\chapter{Desarrollo}
\section{Metodología}
Para el desarrollo de este proyecto se ha optado por una metodología iterativa incremental, siguiendo los principios de desarrollo ágil y permitiendo la adaptación del desarrollo a las necesidades emergentes. Esta metodología se caracteriza por definir un número ``N'' de iteraciones en las que se va entregando por cada una de ellas una versión más completa del proyecto.

Asimismo, no se consideró correcta la elección de otras metodologías ágiles como ``\gls{scrum}'' ya que a pesar de ser un trabajo colaborativo, la línea que se desarrolla en este tfg era independiente.

\section{Fases de desarrollo}
\subsection{Iteración 1}
\subsubsection{Análisis}
El primer paso que se tomó para el desarrollo fue contactar tanto con el \gls{itc}, como con el \gls{jbvc} para obtener una idea más concreta de las necesidades inmediatas que se requieren de este proyecto. Nótese que este desarrollo trata del framework que se va a utilizar para desarrollar la solución del \gls{jbvc}, y no es la solución en sí. Sin embargo, se consideró necesaria la interacción con los \gls{stakeholders} para concretar la extensión de la abstracción requerida del framework.

La primera reunión con el \gls{jbvc} se realizó el viernes 06 de febrero dentro del recinto del Jardín Botánico. En esta reunión se concretaron los detalles y requisitos funcionales que se requieren, así como las estructuras de datos ya existentes que requerirían de migración al nuevo sistema.

Debido a que el proyecto ya estaba empezado, se requirió de un análisis previo del código ya escrito así como de los requisitos y exigencias que se esperan de este desarrollo. Una vez observado el estado del desarrollo (Bastante temprano) se destinó la segunda semana de febrero entera a disponer de una base funcional y conectada entre los dos principales componentes en los que este TFG está fundamentado.