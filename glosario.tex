\makeglossaries

\newacronym{itc}{ITC}{Instituto Tecnológico de Canarias}
\newacronym{jbvc}{JBVC}{Jardín Botánico Viera Y Clavijo}
\newglossaryentry{scrum}{
    name=Scrum,
    description={Marco de trabajo (framework) ágil y ligero que ayuda a las personas, equipos y organizaciones a generar valor a través de soluciones adaptativas para problemas complejos}
}
\newglossaryentry{testing}
{
    name=testing,
    description={Proceso de pruebas concretas al software con el objetivo de asegurar el producto y depurar errores}
}
\newglossaryentry{testear}{
  name={testear},
  description={}, 
  see={testing}  
}
\newglossaryentry{latex}
{
    name=latex,
    description={Is a markup language specially suited 
    for scientific documents}
}

\newglossaryentry{maths}
{
    name=mathematics,
    description={Mathematics is what mathematicians do}
}
\newglossaryentry{nextgendem}
{
  name={NextGenDem},
  description={Proyecto de software web modular para la gestión de datos bioinformáticos
   desarrollado por ITC}
}
\newglossaryentry{ngtgl}{
  name={NextGenDem-GUI-LIB},
  description={Librería frontend desarrollada para facilitar el desarrollo de aplicaciones
  web basadas en NextGenDem}
}
\newglossaryentry{iterinc}{
  name={Metodología Iterativa Incremental},
  description={Metodología de desarrollo de software basada en incrementos iterativos,
  permitiendo adaptarse a cambios y nuevas funcionalidades de manera gradual. En cada cuanto de tiempo
  o iteración se entrega una versión funcional del software con mejoras y nuevas características y 
  al finalizar cada iteración se analiza el estado del proyecto y las necesidades emergentes
  para planificar el siguiente incremento.}
}
\newglossaryentry{framework}{
  name={Framework},
  description={Conjunto de herramientas, bibliotecas y convenciones que proporcionan una estructura base
  para el desarrollo de aplicaciones software. Un framework define una arquitectura común y facilita la
  implementación de funcionalidades, reduciendo la necesidad de escribir código desde cero y promoviendo
  la reutilización, la organización del proyecto y el cumplimiento de buenas prácticas. Su objetivo es
  acelerar el desarrollo y mejorar la mantenibilidad y escalabilidad de las aplicaciones.}
}
\newglossaryentry{frontend}{
  name={Frontend},
  description={Parte de una aplicación software que interactúa directamente con el usuario final.
  Se encarga de la presentación visual, la experiencia de usuario y la interacción mediante interfaces
  gráficas, utilizando tecnologías como HTML, CSS y JavaScript. Su objetivo es ofrecer una comunicación
  eficaz entre el usuario y el sistema.}
}
\newglossaryentry{backend}{
  name={Backend},
  description={Parte de una aplicación software responsable del procesamiento de la lógica de negocio,
  la gestión de datos y la comunicación con bases de datos y servicios externos. Opera de forma no visible
  para el usuario final y garantiza el correcto funcionamiento interno del sistema mediante servidores,
  APIs y mecanismos de seguridad.}
}
\newglossaryentry{jardincanario}{
name={Jardín Botánico Viera y Clavijo},
description={Institución de conservación e investigación botánica situada en Gran Canaria, considerada el mayor jardín botánico de España. Fundado en 1952 por Eric Ragnor Sventenius, su misión principal es la preservación y el estudio de la flora endémica de la Macaronesia y las Islas Canarias en un entorno seminatural}
}
\newglossaryentry{stakeholders}{
    name={Stakeholders},
    description={Conjunto de individuos u organizaciones que afectan o se ven afectados por el desarrollo de un sistema software. Sus intereses, requisitos y restricciones influyen en las decisiones técnicas y organizativas del proyecto, siendo fundamentales para garantizar el éxito y la aceptación del producto final.}
}
